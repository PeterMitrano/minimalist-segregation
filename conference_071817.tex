\documentclass[conference]{IEEEtran}
\IEEEoverridecommandlockouts
% The preceding line is only needed to identify funding in the first footnote. If that is unneeded, please comment it out.
\usepackage{cite}
\usepackage{amsmath,amssymb,amsfonts}
\usepackage{algorithmic}
\usepackage{graphicx}
\usepackage{textcomp}
\def\BibTeX{{\rm B\kern-.05em{\sc i\kern-.025em b}\kern-.08em
    T\kern-.1667em\lower.7ex\hbox{E}\kern-.125emX}}
\begin{document}

\title{Clustering Heterogeneous Objects with Robots That Do Not Compute}

\author{\IEEEauthorblockN{Jordan Burklund}
\IEEEauthorblockA{\textit{Worcester Polytechnic Institute} \\
Worcester, MA \\
jsburklund@wpi.edu}
\and
\IEEEauthorblockN{Michael Giancola}
\IEEEauthorblockA{\textit{Worcester Polytechnic Institute} \\
Worcester, MA \\
mjgiancola@wpi.edu}
\and
\IEEEauthorblockN{Peter Mitrano}
\IEEEauthorblockA{\textit{Worcester Polytechnic Institute} \\
Worcester, MA \\
pdmitrano@wpi.edu}
}

\maketitle

\begin{abstract}
This document is a model and instructions for \LaTeX.
This and the IEEEtran.cls file define the components of your paper [title, text, heads, etc.]. *CRITICAL: Do Not Use Symbols, Special Characters, Footnotes, 
or Math in Paper Title or Abstract.
\end{abstract}

\begin{IEEEkeywords}
component, formatting, style, styling, insert
\end{IEEEkeywords}

\section{Introduction}
This document is a model and instructions for \LaTeX.
Please observe the conference page limits. 

\section{Proposed Work}

\section{Proposed Experiments and Expected Outcomes}

\section{Weekly Schedule}
Test Citation \cite{salek2013hotspotting}


% MIKE: Changed the original citation format since this is way easier to manage. It outputs the same way so it shouldn't be a problem.
\bibliography{references.bib}
\bibliographystyle{unsrt}

% \begin{thebibliography}{00}
% \bibitem{b1} G. Eason, B. Noble, and I. N. Sneddon, ``On certain integrals of Lipschitz-Hankel type involving products of Bessel functions,'' Phil. Trans. Roy. Soc. London, vol. A247, pp. 529--551, April 1955.
% \bibitem{b2} J. Clerk Maxwell, A Treatise on Electricity and Magnetism, 3rd ed., vol. 2. Oxford: Clarendon, 1892, pp.68--73.
% \bibitem{b3} I. S. Jacobs and C. P. Bean, ``Fine particles, thin films and exchange anisotropy,'' in Magnetism, vol. III, G. T. Rado and H. Suhl, Eds. New York: Academic, 1963, pp. 271--350.
% \bibitem{b4} K. Elissa, ``Title of paper if known,'' unpublished.
% \bibitem{b5} R. Nicole, ``Title of paper with only first word capitalized,'' J. Name Stand. Abbrev., in press.
% \bibitem{b6} Y. Yorozu, M. Hirano, K. Oka, and Y. Tagawa, ``Electron spectroscopy studies on magneto-optical media and plastic substrate interface,'' IEEE Transl. J. Magn. Japan, vol. 2, pp. 740--741, August 1987 [Digests 9th Annual Conf. Magnetics Japan, p. 301, 1982].
% \bibitem{b7} M. Young, The Technical Writer's Handbook. Mill Valley, CA: University Science, 1989.
% \end{thebibliography}

\end{document}
