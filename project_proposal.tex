\documentclass[conference]{IEEEtran}
\IEEEoverridecommandlockouts
% The preceding line is only needed to identify funding in the first footnote. If that is unneeded, please comment it out.
\usepackage{cite}
\usepackage{amsmath,amssymb,amsfonts}
\usepackage{algorithmic}
\usepackage{graphicx}
\usepackage{textcomp}
\def\BibTeX{{\rm B\kern-.05em{\sc i\kern-.025em b}\kern-.08em
    T\kern-.1667em\lower.7ex\hbox{E}\kern-.125emX}}
\begin{document}

\title{Clustering Heterogeneous Objects with Robots That Do Not Compute}

\author{\IEEEauthorblockN{Jordan Burklund}
\IEEEauthorblockA{\textit{Worcester Polytechnic Institute} \\
Worcester, MA \\
jsburklund@wpi.edu}
\and
\IEEEauthorblockN{Michael Giancola}
\IEEEauthorblockA{\textit{Worcester Polytechnic Institute} \\
Worcester, MA \\
mjgiancola@wpi.edu}
\and
\IEEEauthorblockN{Peter Mitrano}
\IEEEauthorblockA{\textit{Worcester Polytechnic Institute} \\
Worcester, MA \\
pdmitrano@wpi.edu}
}

\maketitle

\begin{abstract}
\end{abstract}

\begin{IEEEkeywords}
  swarm, robotics, object clustering
\end{IEEEkeywords}

\section{Introduction}
In our study of swarm robotics, we have read research on the clustering of objects with swarm robots.

\section{Proposed Work}
We plan to extend existing work to the case where there are two or more classes of objects that need to be seperated. We will first begin by reproducing the results of the other paper...

\section{Proposed Experiments and Expected Outcomes}

\section{Weekly Schedule}
Test Citation \cite{salek2013hotspotting}

\bibliography{references.bib}
\bibliographystyle{unsrt}

\end{document}
